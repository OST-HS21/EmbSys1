\section{Interrupt Concepts \refskript{7.1, 7.2}}
A Interrupt is a signal indicating the occurrence of an event that needs immediate CPU attention.
There exist different types of Interrupts:
\begin{itemize}
	\itemsep-.5em 
	\item Hardware triggers
	\item Software triggers
	\item CPU exceptions
\end{itemize}

\begin{lstlisting}[language=c]
PxDIR &= ~0x01;	// Set Px.0 as input
PxIES |=  0x01; // Select interrupt edge high-to-low
PxIE  |=  0x01;	// Enable interrupt on SW1

PxDIR |=  0x02;	// Set Px.1 as output
PxOUT &= ~0x02;	// Clear LED on Px.1
...
__interrupt void portx_ISR(void) {
	PxOUT ^=  0x02;
}
\end{lstlisting}\vspace{-20px}

\subsection{Maskabe vs. Nonmaskable \refskript{7.1.2}}
\textit{Maksable}: Can be blocked (masked), trough flags. Most common type of interrupt. Disabled upon RESET.

\textit{Non-maskable} interrupts (NMI): Cannot be masked, thus are always served. Reserved for system critical events

Also see \refskript{7.2.2} and \refskript{7.2.3}


\subsubsection{Interrupt Flow}
\includegraphics[width=.6\columnwidth, center]{"InterruptFlow"}

Interrupt kann über intrinsische Funktionen eingestellt werden.
Alternativ mit enable\_interrupt() funktion.


\subsubsection{Interrupt Service Sequence \refskript{7.1.3}}
PC speicher den aktuellen ort im code bevor der Interrupt ausgeführ wird.
SR wird auf dem stack gespeichert

\subsubsection{Interrupt Identification Methods \refskript{7.1.4}}
\textit{Non-vectored systems (1)} don't know, what triggered the interrupt and have to identify the interrupt source within the ISR.

\textit{Vectored interrupts (2)} Have an ID for each interrupt type an can jump directly to the desired ISR.

\textit{Auto-vectored interrupts (3)} have predefined addresses / fix vectors for the different interrupts.

\subsubsection{Interrupt Priority Handling \refskript{7.1.5}}
(1) \textit{Polling order of SRQ flags} decides priority (if else if define the interrupt priority)

(2) \textit{Daisy Chain-based Arbitration} defines the priority through hardwired logic. Used by the MSP430.

(3) \textit{Interrupt controller-based Arbitration} allows priority configuration by the user.

\subsection{Interrupt Software Design \refskript{7.3}}
\begin{lstlisting}[language=c]
// main.c
int main(void){
	hal_gpio_init();
	hal_gpio_interruptCallbackFctRegister(
					GPIO_S1,_s1_callbackFct);
}

static void _s1_callbackFct(void){
	...
}
\end{lstlisting}\vspace{-20px}
\begin{lstlisting}[language=c]
// hal_gpio.h
void hal_gpio_init(void);
void hal_gpio_interruptCallbackFctRegister(
			gpioPeripherie_t gpio, pFctHandler pCallbackFct);
\end{lstlisting}\vspace{-20px}
\begin{lstlisting}[language=c]
// hal_gpio.c
void hal_gpio_init(void) { // ... }
void hal_gpio_interruptCallbackFctRegister(
		gpioPeripherie_t gpio, pFctHandler pCallbackFct) {
			_hal_gpio_callbackFct[0] = pCallbackFct;
}

#pragma vector=PORT1_VECTOR
__interrupt void _port1_ISR(void) {
	// ...
	_hal_gpio_callbackFct[0](); // call back application code
}
\end{lstlisting}\vspace{-20px}



