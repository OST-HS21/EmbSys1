\subsubsection{UART \refskript{9.3.4}}
\begin{itemize}
	\itemsep-.5em
	\item Handshaking Lines \refskript{9.3.6}
	\begin{itemize}
		\itemsep-.5em 
		\item \textbf{DSR} Data Set Ready
		\item \textbf{DTR} Data Terminal Ready
		\item \textbf{RTS} Request to Send
		\item \textbf{CTS} Clear to Send
	\end{itemize}
	\item Communication Lines
	\begin{itemize}
		\itemsep-.5em
		\item \textbf{TxD} Transmitted Data
		\item \textbf{RxD} Received Data
	\end{itemize}
\end{itemize}

UART configuration \refskript{9.3.7}


\subsubsection{Baud rate}
Bit rate: Bits per second\\
Baud rate: Symbols per second

\subsubsection{RS-232 \refskript{9.3.8}}
Point to point connection.
The transfer-speed is highly dependent on the transmission line length.

\subsubsection{RS-422}
Bus connection with a maximum of 10 receivers (more with repeaters or buffers).

\subsubsection{RS-485}
Improved characteristics over RS-422 with up to 32 drivers and 32 receivers.

\subsection{Ring Buffer}
A ring buffers works with the FIFO principle.

\begin{lstlisting}[language=C]
// --- Defines
#define _UART_RX_BUF_BITS 5 // Buffer Size: 2^n
#define _UART_RX_BUF_SIZE (0x0001 << _UART_RX_BUF_BITS)
#define _UART_RX_BUF_MASK (_UART_RX_BUF_SIZE - 1)
	
// --- Variables
volatile struct { // RX Circular buffer
	uint8_t buf[_UART_RX_BUF_SIZE];
	uint16_t in;
	uint16_t out;
} rxBuffer;

uint8_t hal_uart_rxChar(uint8_t *const pdataRx) {
	uint8_t nrOfChar = 0x00;
	
	UCA0IE &= ~0x0001; // temporary disable Rx-ISR
	// check rx-Buffer state
	if (rxBuffer.in == rxBuffer.out) {
		// rx-Buffer is empty
		UCA0IE |= 0x0001;	// enable Rx-ISR
		nrOfChar = 0x00;
	} else { // rx-Buffer contains data
		// get Character from rx-Buffer
		*pdataRx = rxBuffer.buf[rxBuffer.out++];
		// wrap-around index
		rxBuffer.out &= _UART_RX_BUF_MASK;
		
		UCA0IE |= 0x0001; // enable Rx-ISR
		nrOfChar = 0x01;
	}
	return nrOfChar;
}
\end{lstlisting}